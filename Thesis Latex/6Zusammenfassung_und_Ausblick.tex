\section{Zusammenfassung}

In dieser Arbeit wurde die Forschungsaufgabe bearbeitet, ein Konzept für die Automatisierung von Skills Labs mithilfe von mobilen Robotern zu entwickeln. Dies ist auf Grundlage von Experteninterviews und einer Literaturrecherche erfolgt, welche die Anforderungen des Konzepts vorgegeben haben.

Es wurde identifiziert, dass ein solches Automationssystem in der Lage sein muss, sich mit Geräten wie mobilen Robotern zu verbinden und an diese Befehle zu geben, die den Anweisungen der Nutzer entsprechen. Dazu sollen die Nutzer in der Lage sein, Szenarien zu erstellen, die vom System autonom umgesetzt werden, ohne dass die Roboter von Menschen überwacht werden müssen. Damit dies reibungslos funktioniert, muss es ein Multi-User Konzept geben, welches den Zugriff von mehreren Nutzern ohne die Entstehung von Kollisionen erlaubt.

Auf Grundlage dieser Anforderungen wurde ein Konzept entwickelt, welches um die vier Eckpfeiler User Interface, Logistik, Planung und Navigation herum aufgebaut ist. Das User Interface nimmt die Anweisungen der Nutzer entgegen und zeigt Statusmeldungen des Systems an. Die Logistik verwaltet den Bestand an Gegenständen und Möbeln, welche in den Szenarien verwendet werden. Die Planung verwaltet im Gegenzug die Szenarien selbst und plant die Ausführung dieser. Dazu weist sie auch die Logistik an, Gegenstände zu reservieren, damit diese zum Ausführungszeitpunkt verfügbar sind. Die Navigation schließlich ist für die Ausführung dieser Szenarien verantwortlich. Sie kommuniziert mit den Robotern und gibt ihnen Navigationsanweisungen, um die Szenarien in der Praxis umzusetzen.

Basierend auf dem Konzept wurde dann ein Prototyp des Systems gebaut, um die Machbarkeit zu belegen. Dieser umfasst die ersten drei Eckpfeiler, welche die nutzerseitigen Teile des Systems bilden. Es wurde gezeigt, dass ein solches System im Rahmen der gestellten Anforderungen auch mit begrenzten Ressourcen umsetzbar ist. Es wurde allerdings auch festgestellt, dass die Implementierung der Navigation einen größeren Arbeitsaufwand erfordert, als im Rahmen dieser Bachelorarbeit umgesetzt werden kann.

In einer abschließenden Diskussion wurden die zur Implementierung nötigen Schritte festgestellt. Es wurde dargestellt, welche Komponenten noch zu einer Ausführung in der Praxis fehlen und wie diese entwickelt werden können. Dabei wurden die Herausforderungen einer Implementierung in neu gebauten und bestehenden Skills Labs identifiziert und ein Ausblick auf eine Weiterentwicklung des Konzepts gegeben.

\newpage \section{Ausblick}

Die hier vorgestellte Arbeit stellt einen Wegplan für das vorgestellte System dar. Für eine Verwendung des Systems in der Praxis ist weitere Forschungs- und Entwicklungsarbeit vonnöten. Primär muss ein Algorithmus zur Ausführung der Navigation entwickelt werden. Die vorgestellten Ansätze der derzeitigen Forschung bieten hierfür einen Einstiegspunkt an, während die noch nicht umgesetzten Anforderungen die Rahmenbedingungen stellen. Dazu muss eine Benutzeroberfläche entwickelt werden, welche das geforderte Multi-User Konzept umsetzt und einen ergonomischen Zugang zum System gewährt. Auch ist es wichtig, dass bei einer Implementierung in einem konkreten Skills Lab die Voraussetzungen und zu automatisierenden Aufgaben untersucht werden und eine Standort- und Nutzeranalyse durchgeführt wird. Dabei können insbesondere weitere Verwendungsmöglichkeiten des Systems erarbeitet werden, die über den Transport von Möbeln hinausgehen.

Das hier vorgestellte System soll daher nicht als fertige Lösung angesehen werden, sondern als Einstieg in das Forschungsfeld der Automation im Kontext der Life Science Lehre. Da bisher wenige Arbeiten auf diesem Feld durchgeführt wurden, gibt es viele noch nicht betrachtete Blickwinkel, aus denen dies beleuchtet werden kann. Die Modernisierung und Verwendung von digitalen Werkzeugen in der Lehre wird vermutlich von immer größerer Bedeutung sein. Besonders bei der Ausbildung für die Verwendung von cyber-physikalischen Systemen, die in der Praxis von Bedeutung sind, kann die Integration von Automatismen in der Lehre helfen.

Dabei ist es jedoch wichtig, solche Systeme zu identifizieren, die einen echten Mehrwert bieten, um die Entwicklung dieser zu fördern. Da Automatisierung oft mit beträchtlichen Investitionen verbunden ist, ist die Erforschung von kosteneffizienten Lösungen eine zentrale Voraussetzung für die Akzeptanz solcher Werkzeuge. Die hier vorgestellte Arbeit setzt voraus, dass Roboterplattformen verfügbar sind oder angeschafft werden können. Je nach Aufgabe können diese Kosten verursachen, die für Skills Labs nicht tragbar sind. Eine mögliche Forschungsaufgabe wäre neben der Entwicklung der Navigation also auch, kostengünstige Plattformen zu bauen, welche die Anforderungen erfüllen können, um nicht auf vorgefertigte Geräte zurückgreifen zu müssen.

Die vorgestellten Forschungsansätze, auf denen das System aufbaut, stellen nur einen kleinen Teil der aktuell entwickelten Projekte dar. Ein großer Teil der Automatisierungsforschung ist extrem zweckgebunden, da sie in der Wirtschaft zur Lösung spezifischer Probleme stattfindet. Die Erforschung von universellen Lösungen, auf denen in allen Bereichen des Lebens aufgebaut werden kann, sollte bei der Weiterentwicklung dieses Systems im Vordergrund stehen. Das hier vorgestellte Konzept kann, in einem größeren Maßstab, auch auf Warenhäuser in der Wirtschaft, Krankenhäuser und andere Arten des Unterrichts ausgeweitet werden, ohne den Kern des Konzepts zu verändern. Dementsprechend kann eine Weiterentwicklung auch in Anbetracht anderer Einsatzorte erfolgen, wo eine Automatisierung von Bedeutung sein kann.