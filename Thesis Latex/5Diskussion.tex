\section{Erfüllung der Anforderungen}

Die Implementierung, wie sie im vorherigen Kapitel beschrieben ist, legt den Grundstein für ein System, das allen gestellten Anforderungen gerecht werden kann. Um die Lücke zwischen den bisher erfüllten und gestellten Anforderungen zu schließen, ist jedoch noch weiterer Entwicklungsaufwand nötig. Das in Kapitel 3 gestellte Konzept, wenn wie beschrieben entwickelt, ist jedoch in der Lage, diese Lücke zu schließen.

Die Implementierung stellt vor allem die Grundlage dar, auf der die aufwendiger zu entwickelnden Funktionen aufbauen können. So ist die Eingabe von Nutzeranweisungen durch eine Konsoleneingabe mittels des User Interface-Knotens umgesetzt. Die dazu verwendete Funktion 'select\_service' ist dadurch, dass sie die Eingabe in einen Funktionsaufruf umwandelt, auch in der Lage, jede beliebige später hinzugefügte Aktion des Systems zu unterstützen, wenn diese im Knoten vorliegt. Auch die Status-Anzeige liegt in der Form des NodeUIOutput in einer Form vor, die alle bestehenden Aktionen abdeckt und sich beliebig erweitern lässt, um später hinzugefügte Funktionalitäten abzudecken.

Die Möglichkeit, Szenarien und Voreinstellungen zu erstellen, ist durch den Planungsknoten umgesetzt. Die Funktionen 'create\_goal' und 'create\_plan' ermöglichen es einem Nutzer, Szenarien zu erstellen und diese auch vom Knoten in einen Plan zusammenzufügen, falls dies logistisch möglich ist. Die Erstellung von sich wiederholenden Plänen muss in der derzeitigen Implementierung noch händisch erfolgen, die Schnittstelle ist aber so ausgelegt, dass eine Nutzeroberfläche bei einer Planung an sich wiederholenden Terminen diese parallel anfordern kann. Da die Knoten so angelegt sind, dass gleichzeitige Serviceanfragen jeweils einzelne Kommunikationsknoten öffnen, können diese parallel bearbeitet werden und ermöglichen so auch das Anlegen von mehreren Plänen gleichzeitig, sofern die Nutzeranwendung dies erlaubt.

Die Verständlichkeit der Semantik der Nutzereingabe wird dadurch gewährleistet, dass komplexe Prozesse, wie die Reservierung von Gegenständen für Pläne, im Hintergrund erfolgen, sodass nur selbsterklärende Eingaben erforderlich sind, welche nacheinander angefragt werden können. Dadurch kann gleichzeitig sicher gestellt werden, dass die erforderliche Eingabe per Kommandozeile verständlich ist, während eine Erweiterung um eine umfassendere Nutzeroberfläche, die die erforderlichen Eingaben an den Knoten sendet, nahtlos möglich ist. Durch diese Kapselung der Nutzereingabe ist es auch möglich, den Zugang auf das System auf diese Schnittstelle zu beschränken, ohne Funktionalität zu verlieren. Dadurch, dass ROS2 Nachrichten zwischen Knoten verschlüsseln kann, ist das System auch vor unberechtigtem Zugriff geschützt, wenn Teile des Systems auf verschiedenen Komponenten eines Computernetzes ausgeführt werden, wie es bei mobilen Robotern der Fall ist.

\newpage
Eine weiterführende Implementierung des Konzepts darf selbstverständlich diese bereits erfüllten Anforderungen nicht brechen, während sie die noch ausstehenden Anforderungen erfüllen muss. Der zu entwickelnde Navigationsknoten, welcher die Steuerung der verbundenen Geräte erlaubt, muss in der Lage sein, wie die bestehenden Knoten für eine größere Anzahl an Zugriffen skalierbar zu sein. Dazu muss er auch weiterhin die Verwendung der gesicherten Schnittstellen nutzen und die durchgeführten Aktionen verständlich dem Nutzer mitteilen können. Dies ist bei Beibehaltung der bestehenden Kommunikationsschemata ohne weitere Entwicklungsanstrengungen möglich. 

Es wird davon ausgegangen, dass die größte Herausforderung dieses Moduls die autonome Steuerung anhand der Szenarien und Planungsdaten sein wird. Eine Anwendung der Prinzipien der Arbeiten von Reily et al. \cite{silentSwarm} und Yang et al. in \cite{2DPlan} ist in der Lage, einen weitestgehend autonomen Betrieb des Systems zu erlauben, der auch robust gegenüber den Änderungen von laufenden Plänen ist. Eine zentrale Forschungsaufgabe hierbei wird es sein, den sicheren Betrieb in menschlicher Umgebung zu garantieren.

Eine dedizierte Nutzeranwendung, welche Zugriff auf das System gewährt, kann die verbliebenen noch offenen Anforderungen erfüllen. Eine graphische Oberfläche, welche den Prinzipien moderner nutzerzentrierten Designs folgt, kann die Eingabe der Nutzeranweisungen auch für Laien und Erstnutzer zugänglich machen. Hierbei wird weniger Forschungsarbeit als bei der Entwicklung der Navigation erwartet, da die Entwicklung von Nutzeroberflächen als weitgehend erforscht gilt und die benötigten Rahmenbedingungen durch das System bereits gegeben sind. Dabei muss auch hier darauf geachtet werden, dass die bereits etablierten gesicherten Kommunikationsschnittstellen verwendet werden, falls die Nutzeranwendung keine eigenen verwendet. In beiden Fällen muss darauf geachtet werden, dass ein Multi-User Konzept entwickelt wird, das die Verwendung durch Lehrende, Studenten und Administratoren unterstützt.


\newpage  \section{Implementierungsstrategie}

\subsection{Technische Voraussetzungen}

%Server
Um das konzeptionierte System in der Praxis zu verwenden, muss das Skills Lab einige Anforderungen über den üblichen Rahmen hinaus erfüllen. Zuerst ist ein zentraler Server nötig, auf dem es betrieben werden kann. Dabei muss jedoch keine unüblich große Leistung zur Verfügung stehen. Das System muss keine harten Echtzeit-Anforderungen erfüllen, solange die Rechenzeit von Operationen in einem für die Nutzer akzeptablen Rahmen liegen, da alle Prozesse asynchron stattfinden können.

%Internet
Grundvoraussetzung für die Verwendung von mobilen Roboterplattformen ist die Verfügbarkeit eines kabellosen Netzwerks, über das sich die Plattformen mit dem System verbinden können. Dies ist notwendig, um Anweisungen zu erhalten und Statusmeldungen senden zu können. Primär wäre hierfür ein WLAN-Netzwerk geeignet, da dieses eine universelle Kommunikation aller verwendeten Geräte ermöglicht. Durch den geplanten dezentralen Ansatz in der Navigation, in dem die Wegführung auf dem Roboter selbst stattfindet, muss dabei keine vollständige Abdeckung des Skills Labs mit einem Funknetzwerk erreicht werden. Je besser die Funkabdeckung des Skills Labs ist, desto höher ist die Erreichbarkeit von Plattformen und desto besser ist dementsprechend die Möglichkeit für das System, laufende Prozesse zu überwachen.

%Türen und Schwellen
Zusätzlich zu einem verfügbaren Netzwerkzugang müssen die Räume des Skills Labs so aufgebaut sein, dass Roboter sich in diesem frei bewegen können. Hierzu ist besonders der Zugang zu den verschiedenen Räumen wichtig. Da der überwältigende Anteil mobiler Roboter sich auf Rädern fortbewegt, ist das Vermeiden von Schwellen unabdingbar. Dazu müssen Türrahmen hoch und breit genug sein, damit sich die verwendeten Roboter und die zu transportierenden Möbel ohne Schwierigkeiten hindurch bewegen können. Bei der Verwendung von Türen ist es zudem wichtig, dass diese entweder zu Betriebszeiten offen stehen oder sich automatisch öffnen lassen, damit ein Zugang ohne Begleitung von Menschen möglich ist.

%Lageplan
Um den in der Implementierung verwendeten Koordinaten für Gegenstände und Roboter einen Kontext zu geben, muss eine digitale Karte des Skills Labs im Navigationsknoten vorliegen. Die Annahme ist, dass diese Navigationskarte über ein Koordinaten-Netz mit x- und y-Koordinaten verfügt, sodass die Position aller relevanten Bestandteile eindeutig beschrieben werden kann. Dazu zählen insbesondere Gegenstände, zu denen Roboter navigieren können müssen, sowie Positionen in den Simulationsräumen, an denen die Gegenstände platziert werden müssen. In der bisherigen Konzeptionierung wird von einer statischen Karte ausgegangen, welche nicht Personen oder mobile Möbelstücke umfasst.


\subsection{Aufwandsabschätzung}

%Vorwort
Bis zu einer vollständigen Implementierbarkeit des Systems sind neben der Erfüllung der technischen Voraussetzungen noch einige Schritte notwendig. Als Teil der Entwicklung der Navigations- und Nutzerschnittstellen, welche bereits beschrieben wurden, muss eine Nutzer- und Umgebungsevaluation durchgeführt werden, um die fehlenden Komponenten genau an die Skills Labs, welche diese verwenden sollen, anzupassen. Aus dieser kann auch erst eine genaue Aufstellung des Arbeitsaufwands der benötigten Komponenten erstellt werden, jedoch kann bereits eine ungefähre Einordnung erfolgen.

%Navigation
Die Entwicklung des Navigationsknotens und der dazugehörigen Algorithmen stellt aller Voraussicht nach den Inhalt einer weiteren Forschungsarbeit dar. Die Arbeiten von Reily et al. \cite{silentSwarm}, Yang et al. \cite{2DPlan} und De La Puente et al. \cite{assistRobot} geben einen plausiblen Forschungsansatz vor, wurden bisher aber nicht in einem gemeinsamen Algorithmus verwendet. Auch gilt es, die Schnittstelle zur Verbindung mit dem System zu programmieren.

%Installation
Die Inbetriebnahme des restlichen Systems auf einem bestehenden Server stellt dagegen einen vergleichsweise vernachlässigbaren Aufwand dar. Da ROS2 auf den verbreitetsten Betriebssystemen betrieben werden kann und SQLite über Python gleichermaßen auf nahezu allen Plattformen unterstützt wird, kann davon ausgegangen werden, dass die Installation ohne weitere Anpassungen erfolgen kann. Je nach Wahl und der Verfügbarkeit von Roboterplattformen kann hier der Aufwand jedoch höher sein.

%Roboter-Einbindung
Da das Robot Operating System in seiner zweiten Version im Vergleich zu ROS1 relativ jung ist, ist die Zahl der nativ unterstützten Roboter auch vergleichsweise gering. Plattformen, die nur die erste Version des Frameworks unterstützen, können durch eine Übersetzungsschnittstelle in ROS2 verwendet werden, dies ist jedoch mit zusätzlichem Aufwand verbunden. Die Einbindung von Robotern, welche keine der beiden Versionen der Framework unterstützen, stellt sich erfahrungsgemäß als möglich, aber sehr aufwendig heraus. Hier müsste eine eigene Schnittstelle für den Roboter entwickelt werden, welche die ROS-Anweisungen in das vom Roboter unterstützten, und die Daten des Roboters zurück in ein vom System verstandenes Format übersetzt.

%Training
Zuletzt muss auch das Training der Nutzer in Betracht gezogen werden. Da das System dafür konzipiert ist, primär von Lehrenden und Betreuern verwendet zu werden, muss sicher gestellt werden, dass diese es auch voll ausschöpfen können. Dazu müssen Schulungen oder Informationsmaterial zur Verfügung gestellt werden, welche im Rahmen der Nutzerschnittstellen-Entwicklung erstellt werden müssen. Zusätzlich zu den Nutzern benötigt das System auch Administratoren, welche das System warten und verwalten können. Da dies sowohl Software- als auch spezialisierte Hardware-Komponenten in Form von Roboterplattformen umfasst, ist das Training dieser dementsprechend aufwendiger.


\section{Implementierung in Skills Labs}

Die Umsetzung des Konzepts in der Praxis kann sich als aufwendig oder gar unmöglich herausstellen, falls die technischen Herausforderungen in einer bestehenden Einrichtung nicht gegeben sind. Während Komponenten, wie eine Abdeckung durch ein Funknetzwerk, vergleichsweise leicht nachrüstbar sind, sind bauliche Einschränkungen schwerer zu beheben. Bestehende Türrahmen, welche über Schwellen oder zu geringe Breiten verfügen, sind kostspielig zu ersetzen. Auch eine komplett fehlende IT-Infrastruktur nachträglich zu verbauen stellt unter Umständen eine zu große Investition dar, um die Kosten- und Arbeitsersparnis zu rechtfertigen. Eine zentrale Voraussetzung ist auch das Vorhandensein eines Lagers, aus welchem die Simulationsräume bestückt werden können. Dieses benötigt eine gewisse Größe und Zugänglichkeit, welche nicht unbedingt durch bestehende Mehrzweck-Räume abgedeckt werden kann.

Bei der Planung von neuen Skills Labs sind diese Voraussetzungen ungleich einfacher zu erfüllen, da viele der Voraussetzungen keinen großen Mehraufwand bei der Planung darstellen. Die Einbindung von WLAN-Netzen wird in Neubauten als üblich angesehen, da von der Verwendung von Computersystemen in der Lehre ausgegangen werden kann. Auch ist die Bereitstellung eines Lagers während der Planung eines Gebäudes wesentlich einfacher umzusetzen als bei einer Nachrüstung.

Eine Gebäudeplanung nach DIN 18040-1, welche die Gestaltung von öffentlich zugänglichen Gebäuden beschreibt, erfüllt auch die baulichen Voraussetzungen für die Verwendung von Roboterplattformen. Insbesondere die Mindestgröße von Türrahmen, die ausreichende Bewegungsfläche vor Zugängen und die Abwesenheit von Türschwellen spielen hier zentrale Rollen. Auch wird in öffentlichen Gebäuden ohnehin gefordert, dass Türen sich automatisch öffnen lassen, sodass eine Einbindung in ein elektronisches System nur einen geringen Mehraufwand darstellt.

Die Auswahl von Roboterplattformen ist hauptsächlich von den Anforderungen der gegebenen Aufgaben abhängig, daher kann hier keine allgemeine Abschätzung erfolgen. Für den Transport von Möbeln, wie Liegen und Tischen, werden beispielsweise Plattformen, die einem Gabelstapler ähneln können, verwendet, während der Transport von Abfalleimern und kleinen Simulatoren und Verbrauchsgegenständen eine Plattform ähnlich einem Rollwagen bevorzugen würde. Die Wahl und eventuelle Entwicklung der Plattformen sollte daher abhängig von allen anderen Faktoren geschehen.


\newpage \section{Reflektion}

Das Projekt hat sich während der Arbeit daran als umfangreicher als initial erwartet herausgestellt. Es ist von vornherein davon ausgegangen worden, dass die Navigation einen großen Teil des Systems darstellen wird, jedoch wurde in der Projektplanung davon ausgegangen, dass diese im Rahmen dieser Forschungsarbeit implementiert werden kann. Während der Recherchearbeit hat sich allerdings herausgestellt, dass die Entwicklung des Frameworks des Systems allein schon den Umfang der Arbeit ausfüllt. Da dementsprechend die Inbetriebnahme des Systems zum Zeitpunkt dieses Berichts aber nicht möglich ist, kann die Forschungsfrage nur als teilweise beantwortet betrachtet werden.

Die Verwendung von ROS als Grundlage für das System wurde schon in der Vorbereitungsphase als beste verfügbare open-source Lösung identifiziert. Die Implementierung bewies, dass das Framework gut für den gewählten Ansatz geeignet ist und alle Anforderungen erfüllen kann. Die Verwendung von SQLite und Python erwies sich als gleichermaßen erfolgreich, auch wenn es hier viele genauso geeignete Lösungen gegeben hat und der Auswahl keine besondere Bedeutung für das Gesamtkonzept zugesprochen wird. Die Auswahl an Werkzeugen erscheint dementsprechend gelungen.

Die Experten-Interviews während der Recherche haben eine interessante Perspektive des derzeitigen Standes der Automation in Skills Labs aufgezeigt. Es ist initial davon ausgegangen worden, dass bereits ein höherer Grad der Einbindung von technischen Systemen vorhanden ist und auf diesen aufgebaut werden kann. Auch die Diversität der Voraussetzungen und Möglichkeiten der verschiedenen Skills Labs erwies sich als größer als vormals angenommen. Dementsprechend wird es als sehr positiv angesehen, dass sie in dieser Form durchgeführt wurden, da so die Bedeutung eines möglichst universell einsetzbaren und einfach zu implementierbaren Systems identifiziert werden konnte. Dementsprechend wird die Recherche als sehr erfolgreich und impulsgebend betrachtet.

Insgesamt kann die Arbeit als Erfolg betrachtet werden. Auch wenn das eingangs gestellte Ziel eines betriebsbereiten Automationssystems nicht vollständig erreicht wurde, wurden wertvolle Erkenntnisse zusammengetragen und eine umfassende Grundlage gelegt. Es ist gelungen, einen realistischen Ausblick auf die Möglichkeiten und Anforderungen eines solchen Systems darzustellen. Dadurch wird Skills Labs die Möglichkeit gegeben, eine informierte Entscheidung über eine Weiterverfolgung dieses Ansatzes zu treffen.