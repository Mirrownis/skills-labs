\section{Hintergrund}

In vielen Krankenhäusern ist es inzwischen üblich, Prozesse zu automatisieren und die Logistik in Teilen intelligenten Systemen zu überlassen. Dies spiegelt sich jedoch nicht in der Lehre wieder, wo eine breite Palette von Fähigkeiten erlernt werden soll. Gerade in Skills Labs, in denen Abläufe anhand von Simulationen geübt werden, findet sich ein sehr geringer Grad der Automation. Dies führt dazu, dass Simulationen entweder mit einem großen Aufwand für den Aufbau verbunden sind oder nur sehr wenige verschiedene Szenarien dargestellt werden, um den begleitenden Aufwand zu minimieren.

Die meisten Arbeiten bei der Vorbereitung von Simulationsräumen können heutzutage aber in der Praxis von Robotern und Computern erledigt werden. Dazu zählt insbesondere der Aufbau der Simulation, die Verwaltung und das Auffüllen von Verbrauchsgegenständen und die Vorbereitung der Räume selbst. Dabei ist automatische Lagerverwaltung ein weit erforschtes Feld, genauso wie der Transport von Waren in Gebäuden und der Aufbau von vorgefertigten Bauteilen. Auch die Warenwirtschaft ist heutzutage in Unternehmen standardmäßig automatisiert.

Durch eine Reduzierung des Arbeitsaufwands dieser Aufgaben kann Lehrenden mehr Zeit für die tatsächliche Lehre zur Verfügung gestellt werden. Auch bietet eine Automation die Möglichkeit, die Qualität der Lehre zu erhöhen. Dadurch, dass die Vorbereitungszeit verringert wird, können aufwendigere Simulationen durchgeführt werden oder das vorhandene Angebot um eine breitere Spanne von Simulationen erweitert werden.


\section{Zielsetzung}

Forschungsaufgabe dieser Bachelorarbeit ist es, ein Konzept für ein umfassendes Automatisierungssystem in einem Skills Lab zu entwickeln. Das System soll in der Lage sein, Lagerbestände zu verwalten und Simulationen aufzubauen, ohne dass Lehrende daran mitwirken müssen. Es soll von den Lehrenden benutzt werden können, um die Arbeit in Skills Labs zu vereinfachen und Aufgaben an das System delegieren zu können.

Dafür sollen Anwendungsfälle identifiziert werden, in denen ein solches System zum Einsatz kommen könnte, um Anforderungen zu erarbeiten, die das System erfüllen muss. Daraus soll ein Konzept erarbeitet werden, das diese Anforderungen erfüllt und in der Praxis umgesetzt werden kann.


\newpage \section{Methodik}

Um dieses Ziel zu erreichen, wird zunächst eine Analyse des derzeitigen Stands der Technik durchgeführt. Hier wird eine Literaturrecherche durchgeführt, welche Forschungsansätze und bestehende Systeme zusammenträgt. Dazu werden die Bereiche Schwarmrobotik, Lagerlogistik, Krankenhausrobotik und Mensch-Roboter Umgebungen betrachtet. Diese Recherche wird als Einstiegspunkt in die Konzeption verwendet.

Zusätzlich finden Experten-Interviews statt, um die Anwendungsfälle zu identifizieren. Diese Interviews finden mit Verantwortlichen für die existierenden Skills Labs der Universität zu Lübeck sowie anderer Einrichtungen statt. Aus diesen Interviews werden funktionale Anforderungen an das System gestellt, die es zu erfüllen gilt. Auch werden dabei mögliche Feature-Sets sondiert, welche das System erweitern können, um einen größeren Mehrwert zu bieten.

Im Anschluss wird ein Konzept erarbeitet, welches den gestellten Anforderungen gerecht wird. Dazu werden die benötigten Komponenten des Systems erarbeitet und in eine gemeinsame Systemarchitektur integriert. Dieses Konzept wird dann in einer Implementierung realisiert, um die Umsetzbarkeit zu beweisen. Zur Erbringung des Beweises wird eine Evaluation der Implementierung durchgeführt.

In einer anschließenden Diskussion wird schließlich das realisierte Konzept mit den Anforderungen verglichen und ein weiterer Wegplan für das System aufgezeigt. Dies umfasst das Erfassen einer Aufwandsabschätzung und der technischen Voraussetzung und Herausforderungen einer Implementierung.